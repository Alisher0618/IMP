\graphicspath{{Images/}}
\hypersetup{linkcolor=blue, urlcolor=blue}

\section{Popis implementace}

\begin{spacing}{1.5}
    \subsection{Funkčnost}
    \fontsize{14}{14}\selectfont
    Nastavení komunikace s displejem, senzorem a rotačním enkodérem probíhá ve funkcí \textbf{setup()}.
    \\
    Funkce \textbf{loop()} - neustálá smyčka provádějící hlavní operace, a to měření vzdálenosti, získávání nové pozici enkodéru a kontrola, zda nebylo překročeno minimální nebo maximální možná hodnota, zobrazování informace o vzdálenosti a zpoždění na OLED displeji, pokud je vzdálenost mimo rozsah senzoru, vypíše se hlášení \textbf{"Out of range"}. Podrobnější informace o tomto hlášení je v nasledující sekci.
    \\
    Funkce \textbf{switchLanguage()} slouží pro přepinání mezi hlavním menu a menu zvoleného jazyka, kde se ukazuje vzdalenost.
    \\
    Funkce \textbf{checkPosition()} slouží pro kontrolu, jestli došlo ke rotaci hřídele. Metoda \textbf{tick()} může aktualizovat informace o poloze nebo počtu otočení.
    \\
    \\
    \subsection{Nečekané chování}
    \fontsize{14}{14}\selectfont
    Během vývoje bylo zjištěno, že dosah laseru je 1-2 metry. Zároveň ale laser vypisuje nepochopitelnou konstantu \textbf{819.10}, pokud je naměřená vzdálenost vyšší než výpočetní možnosti senzoru, tak jsem se rozhodl přidat hodnotu, která by takovému zmatku pomohla předejít. Po četných testech, které zahrnovaly testování laseru v různých světelných podmínkách, s použitím různých materiálů, od kterých by se laserový paprsek odrážel, byla zvolena hodnota \textbf{101 cm}. Hodnota větší než zvolená konstanta by byla považována za \textbf{"Out of range"}.
    \\
    \subsection{Rozšíření}
    \fontsize{14}{14}\selectfont
    Napadlo mě přidat do projektu pár rozšíření. První dovoluje uživateli zvolit preferovaný jazyk. Po zapuntí mikrokontrolleru, uživatel vidí menu a pomocí enkoderu může vybrat jazyk. Na výběr je čeština a angličtina. Druhé rozšíření zobrazuje na displeji maximální naměřenou hodnotu.
    
\end{spacing}